%%% Template originaly created by Karol Kozioł (mail@karol-koziol.net) and modified for ShareLaTeX use

\documentclass[a4paper,11pt]{article}

\usepackage[T1]{fontenc}
\usepackage[utf8]{inputenc}
\usepackage{graphicx}
\usepackage{xcolor}

\renewcommand\familydefault{\sfdefault}
\usepackage{tgheros}
\usepackage[defaultmono]{droidmono}

\usepackage{amsmath,amssymb,amsthm,textcomp}
\usepackage{enumerate}
\usepackage{multicol}
\usepackage{tikz}
\usepackage{soul}

\usepackage{geometry}
\geometry{total={210mm,297mm},
left=25mm,right=25mm,%
bindingoffset=0mm, top=20mm,bottom=20mm}

\usepackage{hyperref}
\hypersetup{
    colorlinks=true,
    linkcolor=blue,
    filecolor=magenta,
    urlcolor=cyan,
}

\urlstyle{same}

\linespread{1.3}

\newcommand{\linia}{\rule{\linewidth}{0.5pt}}

% custom theorems if needed
\newtheoremstyle{mytheor}
    {1ex}{1ex}{\normalfont}{0pt}{\scshape}{.}{1ex}
    {{\thmname{#1 }}{\thmnumber{#2}}{\thmnote{ (#3)}}}

\theoremstyle{mytheor}
\newtheorem{defi}{Definition}

% my own titles
\makeatletter
\renewcommand{\maketitle}{
\begin{center}
\vspace{2ex}
{\huge \textsc{\@title}}
\vspace{1ex}
\\
\linia\\
\@author \hfill \@date
\vspace{4ex}
\end{center}
}
\makeatother
%%%

% custom footers and headers
\usepackage{fancyhdr}
\pagestyle{fancy}
\lhead{}
\chead{}
\rhead{}
%\lfoot{Assignment \textnumero{} 5}
\cfoot{}
\rfoot{Page \thepage}
\renewcommand{\headrulewidth}{0pt}
\renewcommand{\footrulewidth}{0pt}
%

% code listing settings
\usepackage{listings}
\lstset{
    language=Python,
    basicstyle=\ttfamily\small,
    aboveskip={1.0\baselineskip},
    belowskip={1.0\baselineskip},
    columns=fixed,
    extendedchars=true,
    breaklines=true,
    tabsize=4,
    prebreak=\raisebox{0ex}[0ex][0ex]{\ensuremath{\hookleftarrow}},
    frame=lines,
    showtabs=false,
    showspaces=false,
    showstringspaces=false,
    keywordstyle=\color[rgb]{0.627,0.126,0.941},
    commentstyle=\color[rgb]{0.133,0.545,0.133},
    stringstyle=\color[rgb]{01,0,0},
    numbers=left,
    numberstyle=\small,
    stepnumber=1,
    numbersep=10pt,
    captionpos=t,
    escapeinside={\%*}{*)}
}

%%%----------%%%----------%%%----------%%%----------%%%

\begin{document}

\title{Simple Packtivity Example Using Umbrella}

\author{Elizabeth Caldwell - {\it Center for Research Computing, University of Notre Dame}}

\date{12 April 2017}

\maketitle

\section*{Input File Example: }
This simple example will take the contents of an input file and save them in an output file.
\begin{lstlisting}[label={list:first},caption=packtivity specification: inputfile-test.yml]
process:
  process_type: 'string-interpolated-cmd'
  cmd: 'cat {inputfile} > {outputfile}'
publisher:
  publisher_type: 'frompar-pub'
  outputmap:
    outputfile: outputfile
environment:
  environment_type: 'umbrella'
  image: 'centos6'
  spec_url: '/home/beth/Desktop/DASPOS/fast.umbrella'
\end{lstlisting}
You will notice that the packtivity spec environment refers to a spec called {\bf fast.umbrella}. The {\bf fast.umbrella} file is provided in the repository. You will need to adjust the spec\_url in the packtivity specification to the location of this file on your own machine.
Now you are ready to run the packtivity. Run the following command to execute this packtivity.
\begin{lstlisting}[label={list:first},caption=packtivity-run command]
packtivity-run input-example.yml \
-p inputfile="'{workdir}/myinputfile.txt'" \
-p outputfile="'{workdir}/input-example-outputfile'"
\end{lstlisting}
When the packtivity has finished running, you should see a new file in the directory you ran the command in. This is the outputfile specified in the command and specification. If you open this file, you will see that it has the same text as the given input file.
\\\\This example can be found in the following GitHub repository: \\\url{https://github.com/ecaldwe1/packtivity-with-umbrella-examples/tree/master/input-file-example}
\end{document}
